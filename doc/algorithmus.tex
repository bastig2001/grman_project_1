\section{Algorithmus}

Ein Wahlalgorithmus, welcher sich für ein verteiltes System mit der Topologie eines Ringes eignet, ist der Algorithmus nach 
\textit{Chang und Roberts}\cite{ChangRoberts79}.Dieser ist insbesondere für gerichtete Ringe angebracht, da sämtliche Kommunikation 
nur in eine Richtung ausgeführt wird.


\subsection{Anforderungen und Annahmen}
\label{sec:anf-an}

Die grundlegendsten Anforderungen an ein jeden Wahlalgorithmus in einem verteilten System sind:
\begin{itemize}
    \item Nach Ende der Wahl muss es genau einen Leiter geben.
    \item Nach Ende der Wahl wissen alle Prozesse, wer der neue Leiter ist.
\end{itemize}

Diese Anforderungen kann der Chang-Roberts-Algorithmus grundsätzlich erfüllen. Damit der Algorithmus auch wirklich wie erwartet funktioniert, müssen
noch folgende Annahmen über das System zutreffen\cite{bonakdarpur}:
\begin{itemize}
    \item Der Algorithmus ist dezentral, d.h. jeder Prozess ist in der Lage den Algorithmus, also eine Wahl, zu initiieren.
    \item Jeder Prozess hat eine unter allen Prozessen einmalig vorkommende Kennung, \textit{Id}, und diese \textit{Id}s unterliegen einer Totalordnung. 
    \item Alle Prozesse agieren bei der Wahl lokal nach dem selben Algorithmus.
\end{itemize}

Treffen diese Annahmen zu, so wird der Chang-Roberts-Algorithmus wie erwartet funktionieren 
und dem System zu genau einem Leiter, der allen bekannt ist, verhelfen.

\subsection{Ablauf}
\label{sec:ablauf}

Beim Chang-Roberts-Wahlalgorithmus wird der Prozess mit der größten Id zum Leiter ernannt. Der Ablauf, wie man zu diesem Endstand kommt sieht wie folgt aus:
\begin{enumerate}
    \item Der Ausgangspunkt ist, dass kein Prozess an einer Wahl teilnimmt.
    \item Ein Prozess, der das Fehlen eines Leiters bemerkt, leitet eine Wahl ein. 
          Er stellt sich selber zur Wahl auf, indem er eine Nachricht mit seiner Id als Wahlvorschlag an seinen Nachbarn schickt -- da die Wahl in einem 
          gerichteten bzw. unidirektionalen Ring stattfindet, hat jeder Prozess nur einen Nachbarn, an den er schicken kann, und einen, von dem er empfängt 
          -- und markiert sich selber als Wahlteilnehmer.
    \item Jeder Prozess, der einen Wahlvorschlag erhält und
          \begin{enumerate}
              \item noch kein Wahlteilnehmer ist, markiert sich als Wahlteilnehmer und
                    \begin{enumerate}
                        \item sendet den Wahlvorschlag weiter, wenn seine Id niedriger als die vorgeschlagene ist.
                        \item sendet seine eigene Id als Wahlvorschlag an seinen Nachbarn, wenn diese höher als die vorgeschlagene ist.
                        \item der Vorschlag kann nicht gleich der eigenen Id sein, da jeder Prozess eine einmalige hat (siehe Annahmen \ref{sec:anf-an}).
                    \end{enumerate}
                \item bereits Wahlteilnehmer ist, 
                    \begin{enumerate}
                        \item ignoriert den Wahlvorschlag, wenn dieser niedriger als die eigene Id ist -- dadurch werden überflüssige Wahlen vorzeitig 
                              beendet.
                        \item sendet den Wahlvorschlag weiter, wenn sein Id niedriger als die vorgeschlagene ist -- ein anderer Prozess mit einer höheren Id
                              wird die überflüssige Wahl beenden.
                        \item wird zum neuen Leiter und beginnt die Wahl zu beenden, wenn der Vorschlag gleich der eigenen Id ist.
                    \end{enumerate}
          \end{enumerate}
    \item Wenn ein Prozess zum neuen Leiter gewählt wurde, bedeutet es, dass sein Wahlvorschlag mit seiner Id eine ganze Runde bis zu ihm zurück 
          gemacht hat. Man kann sich also sicher sein, dass er die höchste Id von allen Prozessen im Ring hat. Der Prozess beendet nun seine Wahlteilnahme
          und beginnt die Wahl zu beenden, indem eine Nachricht, dass er die Wahl gewonnen hat, mit seiner Id an seinen Nachbarn schickt.
    \item Jeder nicht gewählte Prozess beendet ebenfalls seine Wahlteilnahme, sobald ihn die Nachricht vom Wahlsieger erreicht hat, und leitet diese weiter.
    \item Hat die Nachricht vom Wahlsieg eine ganze Runde gemacht und ist beim Wahlsieger und Leiter wieder angelangt, leitet dieser sie nicht mehr weiter
          und die Wahl ist beendet.
\end{enumerate}

Sofern die Kommunikation fehlerfrei abgelaufen ist, also alle Prozesse vollständig an der ganzen Wahl teilnahmen und keine Nachrichten verloren gingen,
hat man am Ende ein System, das sich bereits wieder im Ausgangszustand für eine mögliche nächste Wahl befindet -- kein Prozess ist als Wahlteilnehmer 
markiert --, mit einem Leiter und alle wissen wer dieser ist.

Für einen erfolgreichen Wahlablauf benötigt man bei $n$ Prozessen im Ring und einem Initiator im ungünstigsten Fall $3n-1$ Nachrichten je zwischen zwei 
Nachbarn. Dieser ungünstigste Fall tritt ein, wenn der Initiator der Nachbar nach dem prädestinierten Leiter, dem Prozess mit der höchsten Id, ist.
In diesem Fall braucht es $n-1$ Nachrichten, damit der prädestinierte Wahlsieger an der Wahl teilnimmt und sich vorschlägt, $n$ Nachrichten, damit ihn
sein Vorschlag wieder erreicht und er gewählt ist, und weitere $n$ Nachrichten, damit alle anderen auch davon erfahren.

\subsection{Alternativen}

Der Chang-Roberts-Algorithmus ist durchaus nicht der effizienteste Wahlalgorithmus, den man im Ring verwenden kann. So gibt es zum Beispiel
\textit{Franklins Algorithmus}\cite{franklin82} für ungerichtete Ringe und den \textit{Dolev-Klawe-Rodeh-Algorithmus}, welcher versucht Verbesserungen von 
Franklins Algorithmus auf gerichtete Ringe zu bringen\cite{bonakdarpur}.
