\section{Implementierung}

Der grundlegende Chang-Roberts-Algorithmus ist nicht besonders schwer, doch kann man bei der Implementierung einen gewissen Entscheidungsspielraum versuchen
zu erkunden und den Algorithmus auch mit einer sichereren Kommunikation unterstützen.


\subsection{Ring}

Bei der Ring-Klasse und dem konkreten Ring-Objekt handelt es sich im Grunde um einen eher einfachen Builder, welcher die Arbeiterprozesse mit einzigartigen 
Ids anlegt und sie über Zeiger in einer Ringstruktur miteinander verbindet. Weiter ist Ring auch für das starten und stoppen der Arbeiterprozesse 
verantwortlich und bietet mit der Möglichkeit, Wahlen zu initiieren, auch eine grundlegende Schnittstelle für die Benutzung im Programm.


\subsection{Worker}

Jeder Arbeiter hat eine Id, kennt seine Position im Ring und hat auch Verweise auf alle Arbeiter im Ring, sich selbst eingeschlossen.
Dadurch kann ein Arbeiter beim Ausfall seines Nachbars, die Nachrichten an den nächsten Prozess, also seinen neuen Nachbarn, leiten und das System kommt
nicht zum erliegen, da alle Nachrichten bei einem ausgefallenen Prozess hängen bleiben.

Jeder Arbeiter hat seinen eigenen Nachrichtenpuffer, zu welchem er eine Schnittstelle zur Verfügung stellt und von dem er Nachrichten von seinem Nachbarn in 
Gegenrichtung oder von der Ring-Schnittstelle -- in dem Fall wird der Arbeiter stoppen oder eine Wahl initiieren müssen -- erfasst. Je nachdem, um welche
Art von Nachricht es sich handelt und welchen Inhalt sie hat, wird entsprechend gehandelt.

Der Arbeite implementiert den Chang-Roberts-Wahlalgorithmus so wie in Ablauf \ref{sec:ablauf} beschrieben, mit dem Zusatz, dass wenn der noch amtierende
Leiter, falls es ihn noch gibt, einen Wahlvorschlag erhält, dieser nicht nur an der Wahl teilnimmt sonder auch von seiner Position als Leiter zurücktritt.
Sonst könnte es theoretisch zu mehr als einem Leiter kommen. Diese Ergänzung ist allerdings nur nötig, da in diesem Programm eine Wahl nicht erst
eingeleitet wird, wenn ein Leiter fehlt, sondern durch die Wünsche des Benutzer bereits früher.

\subsection{Message Buffer}

Der Nachrichtenpuffer bietet eine Möglichkeit Nachrichten von einem Arbeiter zum nächsten bzw. vom Ring-Controller zu einem Arbeiter zu senden.
Er bietet zwei verschieden Möglichkeiten dafür an, eine Möglichkeit, bei der die Nachricht lediglich sobald als möglich in den Puffer gelegt wird und 
der Sender mit seiner Tätigkeit wieder weitermachen kann, und eine andere Möglichkeit, bei der zusätzlich gewartet wird, bis die Nachricht vom Gegenüber 
entnommen wurde oder zu viel Zeit verstrichen ist. Die Arbeiter verwenden letzteren Weg für die Kommunikation untereinander. Wenn die Wartezeit ausläuft
bevor der Nachbar die Nachricht entnommen hat, wird angenommen, dass er ausgefallen ist.

\noindent\hrulefill\par
\begin{minipage}{\linewidth}
\begin{lstlisting}[mathescape, language=C++, caption=senden und warten]
lock_guard<mutex> assign_and_wait_lck{
    assign_and_wait_mtx
};

unique_lock<mutex> rendezvous_lck{rendezvous_mtx}; 
message_is_taken = false;
rendezvous_lck.unlock();

assign(message);

rendezvous_lck.lock();
return message_taken.wait_for(
    rendezvous_lck, chrono::milliseconds(waittime), 
    [this](){ return message_is_taken; }
);
\end{lstlisting}
\end{minipage}
$$$$
Bei der Funktion \verb|assign_and_wait| bzw. \textit{senden und warten} sieht man, dass durch den Lock gleich am Anfang nur ein Aufrufer auf einmal
sich in dieser Funktion aufhalten darf. Das liegt daran, dass der Aufrufer, nachdem er die Nachricht mit \verb|assign(message)| sendet, unten darauf
wartet, dass die Nachricht entnommen wird. Da bei einer Condition Variable nicht zwingend der Thread aufgeweckt wird, der länger gewartet hat, ist es
zu vermeiden, dass mehrere Aufrufer gleichzeitig warten bzw. zur Überprüfung, ob sie warten müssen, kommen, sonst könnte dem falschen Sender glaubhaft
gemacht werden, dass seine Nachricht bereits entnommen wurde. Außerdem bestünde noch die Gefahr, dass ein Aufrufer \verb|message_is_taken| auf falsch
setzt kurz, nachdem es vom Empfänger auf wahr gesetzt wurde und noch bevor der wartende Aufrufer darauf reagieren konnte. Diese Probleme sind 
in diesem Programm zwar nicht zu erwarten, da es je Puffer eigentlich nur einen Thread gibt, der diese Funktion benutzt. Das ist der vorrangige Nachbar.
Doch da dieser Lock und Mutex in dieser konkreten Implementierung eigentlich nicht benötigt werden, blockieren sie also auch niemanden.

\noindent\hrulefill\par
\begin{minipage}{\linewidth}
\begin{lstlisting}[language=C++, caption={nur senden, ohne warten}]
unique_lock<mutex> buffer_lck{buffer_mtx};
message_assignable.wait(
    buffer_lck, [this](){ return is_empty(); }
);

this->message = message;
message_assigned = true;
message_takable.notify_one();
\end{lstlisting}
\end{minipage}

Bei der Funktion \verb|assign| bzw. \textit{senden} sieht man, dass am Anfang gewartet wird, dass man etwas in den Puffer legen kann, und es danach
in den Puffer gelegt wird. Außerdem kann man in dieser Funktion auch erkennen, dass es der Puffer nur eine Nachricht halten kann. Der Puffer ist lediglich
eine primitive Variable. Die Entscheidung, einen Puffer der Größe 1 zu benutzen, wurde gemacht, weil einerseits erspart man sich damit ein größeres Objekt
und bei der Benutzung stört es nicht wirklich, da es meistens nur einen Sender, den vorrangige Nachbarn, gibt und manchmal auch einen zweiten, den
Ring-Controller, mehr aber nicht.

\noindent\hrulefill\par
\begin{minipage}{\linewidth}
\begin{lstlisting}[language=C++, caption=Nachricht holen]
unique_lock<mutex> buffer_lck{buffer_mtx};
message_takable.wait(
    buffer_lck, [this](){ return message_assigned; }
);

message_assigned = false;
message_assignable.notify_one();

lock_guard<mutex> rendezvous_lck{rendezvous_mtx};
message_is_taken = true;
message_taken.notify_one();

return message;
\end{lstlisting}
\end{minipage}

Die Funktion \verb|take| bzw. \textit{Nachricht holen} verwendet der Empfänger, um sich die Nachricht zu holen. Am Anfang wird überprüft, ob eine
Nachricht vorhanden ist und wenn nicht wird gewartet. Die Nachricht wird dann entnommen und, falls Threads darauf warten, dass sie eine Nachricht in den
Puffer legen können oder dass ihre Nachricht entnommen wurde, werden diese verständigt.


\subsection{Message}

Bei den Nachrichten selbst handelt es sich lediglich um Strukturen, die einen Nachrichtentyp und eventuell dazugehörige Daten enthalten.

